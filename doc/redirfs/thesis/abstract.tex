\begin{titlepage}
	\subsection*{Abstract}
	This thesis focuses on filesystem access control in the Linux kernel. It
	describes all existing methods of filesystem access control and points
	out their problems and limitations. Further it describes in detail Virtual
	Filesystem Switch (VFS) implementation in the Linux kernel and introduces a
	new framework called Redirfs. It is based on the redirection of the VFS
	objects operations and allows for third-party Linux kernel modules, which are
	called filters, to be notified about all events in the VFS layer.

	\vspace{3cm}
	\subsection*{Keywords}
	VFS, framework, filesystem, LSM, RSBAC, Dazuko, Linux, kernel, LKM, filter,
	avgflt, inode, dentry, super block, vfsmount, cache, FiST, overlay filesystem,
	buffer cache, dentry cache, inode cache, Redirfs, process, Linux kernel linked
	lists, fs\_struct, syscall table, interface, user space, kernel space.
\end{titlepage}
