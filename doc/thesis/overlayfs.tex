\chapter{Overlay Filesystems}
\label{lab:overlay}
Overlay filesystem creates new layer over existing native filesystem. The idea is to
create whole filesystem which will be linked with native filesystem. VFS doesn't
provide any special support for overlay filesystems. Overlay filesystem is mounted
over native filesystem and it works with VFS objects which are created by native
filesystem. Overlay filesystems can be stacked on each other. So for example there can
be native filesystem on which can be linked several overlay filesystems. Problem is
that overlay filesystem duplicates all VFS objects (inode, file, dentry) of filesystem
under it (it can be native filesystem or other overlay filesystem). So if three
overlay filesystems are used over one native filesystem, every VFS object (file,
dentry and inode) has four copies in VFS layer. One copy for native filesystem and
three copies for overlay filesystems. Other problem is that some applications need to
be notified about only a few VFS events. For example anti-virus on-access scanner just
need to know when the file is opened, executed or closed. Write whole filesystem only
to get these events is really a lot of work. There is no way how affect order in
which will be overlay filesystem called and of course overlay filesystem can not be
safely unmounted because it can be used by other overlay filesystem.

\section{FiST}
FiST is a project which tries to make creating of overlay filesystems easier. Actually
it is an overlay filesystem generator. If you want to create new overlay filesystem you
only need to implement functions in which you are interested. FiST will generate whole
filesystem for you. Advantage is that FiST describes overlay filesystem in its special
language an provides generators for Linux, FreeBSD and Solaris. It means that the
overlay filesystem define only once and will work all three operating systems.
