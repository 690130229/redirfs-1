\chapter{Dazuko}
\label{lab:dazuko}
Dazuko aims to be a cross-platform device driver that allows applications to control
file access on a system. The current release of Dazuko supports Linux 2.2-2.6,
Linux/RSBAC, and FreeBSD 4/5 kernels. It is mainly used by anti-virus companies for
on-access scanning. First version was developed by H+BEDV Datentechnik GmbH (anti-virus
company from Germany) which was latter released as an open source code under BSD and
GPL licence. Dazuko is in these days maintained by John Ogness. In the following text
will be described Dazuko for Linux kernels.

Dazuko consists from Linux Kernel module called dazuko and user space library called
dazukoio. Communication between module and library is done through
\texttt{/dev/dazuko} char device with major number 33. 

\section{Dazuko Interface}
Dazuko provides well defined interface which allows user space processes to be
notified about some filesystem events. At this moment Dazuko generates following
events: on\_open, on\_close, on\_close\_modified, on\_exec, on\_unlink and on\_rmdir.
Dazuko provides interface for several programing languages like C, Python, Perl or
Java. Here will be briefly described interface for C language.

All functions return zero on success.

\subsection*{Registration}
\subsubsection{\texttt{int dazukoRegister(const char* groupName, const char *mode)}}
Registers application to the Dazuko module. Argument \texttt{grouName} specifies group
name to which will be the application added. This is useful when application runs in
several instances. Dazuko then delivers event which occurred to the first available
application in the group. The second parameter \texttt{mode} specifies the mode in
which the application will interact with Dazuko. There are currently two modes
available, read-only "r" and read-write "r+". The read-only mode allows an application
to receive all accesses but does not give the application an opportunity to decide if
the access should be allowed or not.

\subsection*{Configuration}
\subsubsection{\texttt{int dazukoSetAccessMask(unsigned long accessMask)}}
Sets bit mask of events which will be delivered to the application. Possible are
\texttt{ON\_OPEN, ON\_CLOSE, ON\_CLOSE\_MODIFIED, ON\_EXEC, ON\_UNLINK} and
\texttt{ON\_RMDIR}.

\subsubsection{\texttt{int dazukoAddIncludePath(const char *path)}}
Adds new directory path (including all subdirectories) for which will be events
delivered to the applications.

\subsubsection{\texttt{int dazukoAddExcludePath(const char *path)}}
Removes directory path.

\subsubsection{\texttt{int dazukoRemoveAllPaths(void)}}
Removes all guarded paths.

\subsection*{Access Control}
\subsubsection{\texttt{int dazukoGetAccess(struct dazuko\_access **acc)}}
This is blocking function which waits until event is delivered. Argument \texttt{acc}
than contains all information about this event.

\subsubsection{\texttt{int dazukoReturnAccess(struct dazuko\_access **acc)}}
Application which is registered with "r+" mode has to call this function because
dazuko module waits for answer.

\subsection*{Unregistration}
\subsubsection{\texttt{int dazukoUnregister(void)}}
Unregisters application form dazuko module.


\section{Interaction with Linux Kernel}

\subsection{Replacement of Syscall Table}
For Linux kernels 2.2.x and 2.4.x Dazuko replaces addresses of selected functions in
the syscall table. Advantage of this approach is that Dazuko doesn't need to modify
Linux kernel source code and it is relatively easy to replace functions in the syscall
table. Generally it is not a good idea to modify syscall table. It is work for some
rootkit, not for security module. Imagine that Linux kernel will contain some rootkit
detector. In this case Dazuko will not be able to work. Other problem is that syscall
functions are to high. This means that Dazuko is not able to catch accesses to the
filesystems, for example, from NFS kernel daemon. Dazuko has to replace
\texttt{sys\_open} syscall to be able to generate on\_open event. This means that
it is called every time the file is opened and it has to check if the file is in some
directory paths which were selected by applications.

\subsection{LSM Interface}
For Linux kernels 2.6.x Dazuko uses LSM framework. Is is better then replacing syscall
table, but maintaining security module for LSM is a lot of work. Problem is in LSM's
approach to the modules stacking. LSM framework also doesn't provide way how to catch
on\_close events.

\subsection{RSBAC Interface}
Dazuko uses for Linux kernels 2.6.x aside from LSM framework also RSBAC framework
which solves problem with on\_close events.

\subsection{Overlay Filesystem}
At this moment Dazuko uses three ways how to interact with Linux kernel. Overlay
filesystem should replace all of them. It is being implemented in these days and uses FiST
generator.
